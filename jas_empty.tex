

% DOUBLE SPACE VERSION FOR SUBMISSION TO THE AMS
\documentclass[12pt]{article}
\usepackage{ametsoc,epsfig,rotating,color,amsmath,graphicx,pdflscape}
\usepackage{lineno,color}\linenumbers
\usepackage[titletoc,title]{appendix}
\usepackage{titlesec}
%
\begin{document}
%
%\linenumbers
%\documentclass[10pt]{article}
%\usepackage{ametsoc2col}
\newcommand{\cred}[1]{{\color{red}{ #1}}}
\newcommand{\cgrey}[1]{{\color{lightgray} #1}}
\newcommand{\cblue}[1]{{\color{blue} #1}}

\newcommand{\myabstract}{
This is the abstract of the test branch version  }
%

 


%\begin{document}
\title{\textbf{\large{Title}}}
%
% Author names, with corresponding author information. 
% [Update and move the \thanks{...} block as appropriate.]
%

\author
{Stephan R. de Roode\thanks{Corresponding author address:  Stephan R. de Roode, s.r.deroode@tudelft.nl, Department of Geoscience and Remote Sensing, Delft University of Technology, Delft, The Netherlands.} ,
 Author 2\thanks{\textit{The Beach}}
\\
\\ 
\\ 
\\
\date{Version of \today}
     } 
% The following block of code will handle the formatting of the title page depnding on whether
% we are formatting a double column (dc) author draft or a single column paper for submission.
% AUTHORS SHOULD SKIP OVER THIS... There is nothing to do in this section of code.
\ifthenelse{\boolean{dc}}
{
\twocolumn[
\begin{@twocolumnfalse}
\amstitle

% Start Abstract (Enter your Abstract above.  Do not enter any text here)
\begin{center}
\begin{minipage}{13.0cm}
\begin{abstract}
	\myabstract
	\newline
	\begin{center}
		\rule{38mm}{0.2mm}
	\end{center}
\end{abstract}
\end{minipage}
\end{center}
\end{@twocolumnfalse}
]
}
{
\amstitle
\begin{abstract}
\myabstract
\end{abstract}
\newpage
}
%%%%%%%%%%%%%%%%%%%%%%%%%%%%%%%%%%%%%%%%%%%%%%%%%%%%%%%%%%%%%%%%%%%%%
% MAIN BODY OF PAPER
%%%%%%%%%%%%%%%%%%%%%%%%%%%%%%%%%%%%%%%%%%%%%%%%%%%%%%%%%%%%%%%%%%%%%



\section{Introduction}
Stratocumulus 
   

\subsection{Summary }

%
 %
 %
\begin{figure*}[ht]
\includegraphics[width=40pc]{wolk.pdf}
\caption{Text.}
\label{fig:test}
\end{figure*}
%
%
 

\begin{table*}
\begin{tabular}{ l l l l l l}
\hline
     &          & ASTEX & Fast & Reference & Slow \\           
\hline 
latitude  & $^0$N  &     34 &   25  &   25 & 25  \\
longitude & $^0$W  &     25 &   125  &   125 & 125  \\
date      &        & 13 June  & 15 July & 15 July & 15 July \\ 
Div       & 10$^{-6}$ s$^{-1}$   &  - & 1.9 & 1.86 & 1.84 \\  
z$_{\mathrm{Div}}$ & km & 1.6 & 2  & 2 & 2 \\
\hline       
\end{tabular}   
\caption{Details of the simulations. Div represents the large-scale divergence of the horizontal mean wind velocities, which is constant in time and constant up to a height of z$_{\mathrm{Div}}$, except for the ASTEX case in which the divergence varies with time.}
\label{tab:set_up}
\end{table*}
 
\begin{acknowledgment}
The investigations were done  
\end{acknowledgment}


\titleformat{\section}{\large\bfseries}{\appendixname~\thesection .}{0.5em}{}
 \renewcommand{\theequation}{A-\arabic{equation}}
  % redefine the command that creates the equation no.
  \setcounter{equation}{0}  % reset counter 
\begin{appendices}

\section{A mixed-layer model for the subcloud layer}

The budget equation for 
\end{appendices}

\ifthenelse{\boolean{dc}}
{}
{\clearpage}
\bibliographystyle{ametsoc}
\bibliography{cumulus}

\end{document}
